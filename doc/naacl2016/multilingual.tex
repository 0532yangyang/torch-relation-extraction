\section{Multilingual Relation Extraction with Zero Annotation \label{sec:multilingual}}

The models described in previous two sections provide broad-coverage relation extraction that can generalize to all possible input entities and text patterns, while avoiding error-prone alignment of distant supervision to a corpus. Next, we describe techniques for an even more challenging generalization task: relation classification for input sentences in completely different languages.

Training a sentence-level relation classifier, either using the alignment-based techniques of Section~\ref{seq:dist}, or the alignment-free method of Section~\ref{sec:uschema}, requires an available KB of seed facts that have supporting evidence in the corpus.  Unfortunately, available KBs have low overlap with corpora in many languages, since KBs have cultural and geographical biases. In response, we perform multilingual relation extraction by jointly modeling a high-resource language, such as English, and an alternative language with no KB annotation. This approach provides via transfer learning of a predictive model to the alternative language, and generalizes naturally to modeling more languages. 


Extending the training technique of Section~\ref{sec:uschema} to corpora in multiple languages can be achieved by factorizing a matrix that mixes data from KB and from the two corpora. In Figure~\ref{tab:multilingual-corpora} we split the entities of a multilingual training corpus into sets depending on whether they have annotation in a KB and what corpora they appear in. We can perform transfer learning of a relation extractor to the low-resource language if there are entity pairs occurring in the two corpora, even if there is no KB annotation for these pairs. Note that we do not use the entity pair embeddings at test time: They are used only to bridge the languages during training. To form predictions in the low-resource language, we can simply apply the pattern scoring approach of Section~\ref{sec:uschema}.

In Section~\ref{sec:results}, we demonstrate that jointly learning models for English and Spanish, with no annotation for the Spanish data, provides fairly accurate Spanish AKBC, and even improves the performance of the English model. Note that we are not performing \textit{zero-shot} learning of a Spanish relation extraction model~\citep{zeroshot}. The relations in the target schema are language-independent concepts, and we have supervision for these in English. 

Little work exists on multilingual relation extraction. \citet{faruqui2015multilingual} perform multilingual OpenIE relation extraction by projecting all languages to English using Google translate. However, as explained in Section~\ref{sec:openIE} the OpenIE paradigm is not amenable to prediction within a fixed schema. Further, their approach does not generalize to low-resource languages where translation may be unavailable -- while we use translation dictionaries to improve our results, our method is effective even without this resource.



\subsection{Tied Sentence Encoders \label{sec:tie-words}}
The sentence encoder approach of Section~\ref{sec:encoder} is complementary to our multilingual modeling technique: we simply use a separate encoder for each language.  This approach is sub-optimal, however, because each sentence encoder will have a separate matrix of word embeddings for its vocabulary, despite the fact that there may be considerable shared structure between the languages. In response, we propose a straightforward method for tying the parameters of the sentence encoders across languages. 

Most work on multilingual word embeddings uses aligned sentences from the Europarl dataset~\citep{koehn2005europarl} to align word embeddings across languages~\citep{Gouws2015,luong2015bilingual,hermann2014multilingual}. Others~\citep{mikolov2013,faruqui2014retrofitting} align separate single-language embedding models using a word-level dictionary. \citet{mikolov2013} use translation pairs to learn a linear transform from one embedding space to another.

Drawing on these dictionary-based techniques, we first obtain a list of word-word translation pairs between the languages using a translation dictionary. The first layer of our deep text encoder consists of a word embedding lookup table. For the aligned word types, we use a single cross-lingual embedding. Details of our approach are described in Section~\ref{sec:word-tying}.

